\documentclass{beamer}

\usepackage[utf8]{inputenc}
\usepackage{natbib}
\usepackage{multirow}
\usepackage{booktabs}
\usepackage{float}
\usepackage{bbding}
\usepackage{array}
\usepackage{tikz}
\usepackage{multicol}
\usepackage{lipsum}
\usepackage{tabularx}
\usepackage{amsmath}
\setlength{\columnseprule}{0.4pt}

\let\Tiny\tiny
\beamertemplatenavigationsymbolsempty

\usetheme{Madrid}
\definecolor{Green}{RGB}{43,134,75}
\usecolortheme{lmugreen}
\setbeamercolor{item projected}{bg=Green,fg=white}

\begin{document}
\title[Algebra \& Coumputer Science]{Title}
\author[Ruben Triwari]{Ruben Triwari}
\institute[LMU Munich]{
    % \large
    {\includegraphics[scale=0.032]{images/LMU_Muenchen_Logo.svg.png}} \\
    Center for Information \& Language Processing (CIS), LMU Munich 
}


\begin{frame}
  \frametitle{Formal languauges and automatas}
  Let $\Sigma$ be an Alphabet and $U,V \subset \Sigma^{*}$ languages.\\
  \vspace*{10px}
  \noindent\begin{tabularx}{\textwidth}{@{}X|X@{}}
    \text{Lanaguage as a formal series:}
    \[ L = \sum (L,w)w\]
    \[ 
      \leftrightsquigarrow  L = 
      \bigcup_{w \in \Sigma^{*}}
      \underbrace{(L,w)}_\text{0 or 1}
      \{w\} 
    \]
    & 
    \text{Language defined in FSK:}
    \[  L \subset \Sigma^{*}\] 

  \end{tabularx}

  \noindent\begin{tabularx}{\textwidth}{@{}X|X@{}}
    \text{Defining plus and multiplication:}
    \[ U + V = \sum ((U,w) + (V,w))w \]
    \[
      U \cdot V = \sum ((U,s)(V,t))w, w = st
    \]
    & 
    \text{Operations we now out of FSK:}
    \[  U \cup V = \{ w \text{ } | \text{ } w \in U \lor w \in V\}\] 
    \[ U \cdot V = \{\ st \text{ } | \text{ } s \in U \land t \in V\}\]

  \end{tabularx}
  \end{frame}

\begin{frame}
  \frametitle{Formal languauges and automatas}
  $\rightsquigarrow$ With these defintions of $+$ and $\cdot$ all languages
  with the alphabet $\Sigma$ form an Algebra. \\
  $\rightsquigarrow$ We can then define automatas with matrices 
  $M \in (\Sigma^{*})^{n \times n}$\\
  $\rightsquigarrow$ With that we can prove Kleene's theorem with matrix
   multiplication:
   \[ L \text{ regular language} 
   \iff  L \text{ is accepted by a finite automata}\]
  $\rightsquigarrow$ Finally, an Overview over weighted automatas and formal power series
\end{frame}


\bibliography{references}
\bibliographystyle{acl_natbib}
\end{document}
